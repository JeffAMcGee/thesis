
\section{Evaluation}
We evaluate the system against several baseline implementations, and evaluate
how the performance of the system varies with the number of contacts a user has
and how many FriendlyLocation chooses to use.

The geo-located users we used to do the evaluation are less concerned about the
privacy of their location information than the average twitter user.
As a result, they tended to give better information in the location field of
their user profile than the average twitter user.
FIXME: we added noise instead
For this evaluation we choose to ignore the contents of the geo-located user's
location fields.
It is likely that better results could be achieved in practice
by using the user's reported location a significant factor in the maximum
likelihood estimation.

We evaluated the FriendlyLocation system against FIXME of the FIXME users from
the evaluation group.
FIXME users from this set were removed because none of their friends or
followers had decodable locations.
(When the crawler initially created the evaluation set it filtered out users
with both zero friends and zero followers.)

We investigate several implementations of the FriendlyLocation system:
\begin{description}
\item[Simple] This system treats all contacts equally---it ignores their
predicted location error and any information about what type of contact it is.
Users are selected randomly and it uses the same parameters for all the users.
\item[Location Error] This shows the value of calculating the median location
error. Users are selected randomly.
\item[Relationship Types] This shows the value of treating contacts differently
based on the type of contact.
Users who are more likely to live near the target user, such as reciprocal
friends, are chosen first.
\item[Full] This is the system described in the previous section. It selects
users based on relationship type, uses the parameters from the training, and
uses predicted location error to evaluate the quality of the contact.
\item[FIXME] Add some more here
\end{description}

In addition to using the FriendlyLocation system as described in the previous
section, we created a few baseline location predictors:
\begin{description}
\item[Median] Finds the median of the latitude and the median of the longitude
of 100 randomly selected contacts.
\item[Mode] Finds the most common location of 100 randomly selected contacts,
and breaks ties by picking one randomly.
\item[Backstrom] FIXME
\item[Omniscient] Finds the contact that is closest to the user from 100 users
selected by the same method used in the Full system.
\end{description}

FIXME: the rest of this section is mostly wrong

Figure~\ref{fig:FinalResults} shows a comparison to the results from the 
different predictors.
Our Full FriendlyLocation system predicts the location within 25 miles FIXME\% of
the time.
Unfortunately, when the predictor system is wrong, it can be very wrong; often
when it would pick not just the wrong city, but the wrong side of the
country.
Both using location error and relationship type improved the predictor.  The
simplified version of FriendlyLocation located users within 25 miles 54\% of
the time which is still better than the either of the baseline predictors.
The omniscient predictor demonstrates that there is still room for improvement
which we discuss in the final section.

We investigated the effect of increasing numbers of contacts on the quality of
the results.
Before removing contacts without location information, we sorted the users into
groups based on the number of contacts the had.
Next, we ran the FriendlyLocation predictor against each of them.
Figure~\ref{fig:LulResults} shows the results of prediction.
The quality of the results from the predictor was lower for users with fewer
than 50 contacts; however, as contacts increased beyond that, it had no
significant effect on the results.


The final step of the evaluation is to investigate if FriendlyLocation would
work with a smaller number of profiles than 100.
For all of the users with more than 200 contacts, we sorted their contacts in
order of decreasing probability that they were local.
Then we ran FriendlyLocation on the top \(n\) contacts for
\(n\in (5,10,15,25,100)\).
The results shown in Figure~\ref{fig:TopResults} for 25 contacts are almost as
good as the results for 100 contacts. This means that our algorithm only needs
location information for around ten users to make reasonable predictions.



