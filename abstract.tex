%%%%%%%%%%%%%%%%%%%%%%%%%%%%%%%%%%%%%%%%%%%%%%%%%%%
%
%  New template code for TAMU Theses and Dissertations starting Fall 2012.
%  For more info about this template or the
%  TAMU LaTeX User's Group, see http://www.howdy.me/.
%
%  Author: Wendy Lynn Turner
%	 Version 1.0
%  Last updated 8/5/2012
%
%%%%%%%%%%%%%%%%%%%%%%%%%%%%%%%%%%%%%%%%%%%%%%%%%%%
%%%%%%%%%%%%%%%%%%%%%%%%%%%%%%%%%%%%%%%%%%%%%%%%%%%%%%%%%%%%%%%%%%%%%
%%                           ABSTRACT
%%%%%%%%%%%%%%%%%%%%%%%%%%%%%%%%%%%%%%%%%%%%%%%%%%%%%%%%%%%%%%%%%%%%%

\chapter*{ABSTRACT}
\addcontentsline{toc}{chapter}{ABSTRACT} % Needs to be set to part, so the TOC doesnt add 'CHAPTER ' prefix in the TOC.

\pagestyle{plain} % No headers, just page numbers
\pagenumbering{roman} % Roman numerals
\setcounter{page}{2}

\indent With the rise of volunteered user location information through services
like Foursquare, Google Latitude, and Facebook Places, we face unprecedented
access to the real-world connections among millions of social media users.
%
This location information is increasingly incorporated into social media for
providing localized content, location-aware recommendations, and other
geo-spatial enabled services.
%
In this paper, we investigate the relationship between the strength of the tie
between a pair of users, and the distance between the pair through an
examination of over 100 million geo-encoded tweets and
over 73 million user profiles collected from \temp{information publicly
posted to} Twitter.
%
Based on this investigation, we identify several factors such as number of
followers and how the users interact that can strongly reveal the distance
between a pair of users, and use these factors to train a decision tree to
distinguish between users who are likely to live nearby and pairs of users who
are likely to live in different areas.
%
We use the results of this decision tree as the input to a maximum likelihood
estimator to predict a user's location.
%
We find that this proposed method significantly improves the results of
location estimation relative to a state-of-the-art technique.
%
Our system reduces the average error distance for 80\% of Twitter users from 40
miles to 21 miles using only information from the user's friends and
friends-of-friends.

\pagebreak{}
