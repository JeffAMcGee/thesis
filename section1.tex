\ifdefined\THESIS
    \pagestyle{plain} % No headers, just page numbers
    \pagenumbering{arabic} % Arabic numerals
    \setcounter{page}{1}
    \chapter{\uppercase {Introduction and Related Work}}
\else
\fi

\section{Introduction}

FIXME: rewrite this whole file


Online social networks are increasingly getting ``geographical.'' Millions of
online
social network users are voluntarily sharing their location through location
sharing
services like Foursquare, short text status updates
through real-time microblogs, and photos in online image hosts. This location
information is increasingly being
incorporated in social media for providing localized content, location-aware
recommendations, and other geo-spatial enabled services. The access to the
trails and connections among millions of social media users provides an
unprecedented opportunity to study the relationship between the geographical
properties of social media users and the social relations connecting them.

With high-quality geographic information, online services become more useful to
the local community.  Online services can bring to light discussions about what
the city council voted on last night.  Online services can identify, and react
to, the crowd of people gathering for the big football game.  Online services
can do a better job suggesting friends.  Online services can reconnect users
with the community that the users neglected when they started spending their
time online.

Geographic information is also useful for researchers.  In the past few years,
there has been growth in research into topics such as geographical user
clustering and location-based social networks.  As the accuracy of geographic
information improves, it will open up new research opportunities.

In this paper, we investigate the interplay of distance and tie strength
through an examination of FIXME million geo-encoded tweets and FIXME million user
profiles collected from Twitter. Concretely, we investigate the relationship
between the strength of the tie between a pair of users and the distance
between the pair.  We identify several factors -- including following,
mentioning, and actively engaging in conversations with another user -- that
can strongly reveal the distance between a pair of users.

FIXME: bimodal? The US data is bimodal, but the world data is not. Should we
discuss this? Most other related papers only look at inside US.

We find a bimodal distribution in the distance between Twitter users, with one
peak around 10 miles from people who live nearby, and another peak around 2500
miles, further validating Twitter's use as both a social network (with
geographically nearby friends) and as a news distribution network (with very
distant relationships).  Figure~\ref{fig:EdgeTypes} gives a preview of this
bimodal distribution for four basic types of relationships.  The red curve
represents a strong, reciprocal friendship whereas the cyan line represents a
relatively weak relationship. The red curve has more users at a distance of ten
miles than the cyan line, indicating that reciprocal friendships are more
likely to be local.

Based on these findings, we propose and evaluate a framework for predicting
location for users based on an analysis of the user's social network,
which has great significance for augmenting traditional social media and
enriching location-based services with more refined and accurate location
estimates.  FriendlyLocation correctly locates FIXME\% of users within 25 miles.

\section{Related Work}

%FIXME: Mr. Cheng wrote some of this.

The study of geographical properties of online social media users has drawn
intensive attention in recent years.  Characterizing network properties in
relation to local geography has been studied in \cite{yardi2010tweeting}.
Lindqvist \cite{lindqvist2011m} analyzed how and why people use location
sharing services, and discussed the privacy issues related to location sharing
services.

Scellato et al. \cite{scellato2011socio} used data from three location based
social networks to investigate the relationship between distance and location,
and they show that connections are not purely caused by geographical or social
factors.  They investigated two random models where they shuffle the user
locations and another where they shuffle the social connections and investigate
what happens to the user location.  They show that if a user has more
connections, then their friends tend to be further away.  They also found that
longer connections are equally likely to be part of a social triangle as
shorter connections.

We know of two previous systems that attempted to estimate the location of
users on social networking sites.  In the first, Facebook researchers analyzed
the physical distance between Facebook users' social relations and utilized the
locations of a user's friends' to predict the user's geographical location
\cite{backstrom2010find}.  They picked users who entered their street address
in the location field of their profile and used this to predict the location of
their friends using maximum likelihood estimation.  However, the information
they used is generally not available outside of Facebook, since people rarely
make their street address public.  Cheng et al. \cite{cheng2010you} propose a
content-based system for locating users on Twitter. They find words that are
highly concentrated in specific regions and build a model to calculate the
probability that a user lives at a location.

FIXME: add new papers here - especially the twitter ones

Gilbert \cite{gilbert2009predicting} shows how the strength of a tie can be
predicted by interaction patterns.  They had volunteers rate the strength of
several friendships and then analyzed the activity between the users can
predict the strength of the friendship.  We argue that tie strength is also
correlated with physical proximity.

There has also been some research into specific aspects of Twitter that is
relevant to our work.  User behavior with regard to the location field in
Twitter user profiles has been studied in \cite{hecht2011tweets}.  One of the
early investigations into Twitter usage by Java \cite{java2007we} showed that
people use twitter for chatter, conversations, sharing information, and
reporting news.  This mixture of uses fits well with the bimodal distribution
of friends and followers that we propose.

Finally, Cranshaw et al. \cite{cranshaw2010bridging} solved the inverse of the
problem we are investigating: they predicted the existence of a social network
tie given precise location information from laptops and cell phones. They used
a collection of features of when users are co-located.  Together, these efforts
have begun to lay a foundation for the study \textit{geo-social media}.

