\ifdefined\THESIS
    \pagestyle{plain} % No headers, just page numbers
    \pagenumbering{arabic} % Arabic numerals
    \setcounter{page}{1}
\else
\fi

\flchap{Introduction}

Location-based social media is widespread, with the adoption of voluntary
user-based location sharing via ``check-in'' services like Foursquare,
geo-tagged posts on Twitter, and photos shared on Flickr and Instagram.
%
These services allow users to annotate their activities with a location field,
ranging from a broad descriptor like ``New York'' or ``USA'' to an extremely
granular latitude/longitude pair  derived from the GPS capabilities of modern
smartphones.
%
Millions of users have already adopted these location sharing services,
providing an unprecedented \textit{geographical perspective} on the trails and
connections among millions of social media users.

By investigating the interplay between geography and social media, both
researchers and practitioners are enabling new applications that leverage
location.
%
Location information is increasingly incorporated into social media for
providing localized content, location-aware recommendations, and other
geo-spatial enabled services.
%
For example, researchers have begun investigating techniques to cluster users
based on their revealed geographic patterns \cite{scellato2010distance},
automatically deriving location-based social networks, and improving friend
suggestion based on geographic proximity \cite{cranshaw2010bridging},
\cite{wang2011human}, \cite{crandall2010inferring}.
%
Localized social media activity -- like Twitter discussions about a recent city
council vote -- can be automatically detected and directed to interested
parties.
%
Crowds of co-located people can be identified and related support services can
be directed toward them -- in the case of a fire or emergency, resources may be
more smartly targeted to the affected regions.

%With high-quality geographic information, online services become more useful to
%the local community.
%%
%Online services can bring to light discussions about what the city council
%voted on last night.
%%
%Online services can identify, and react to, the crowd of people gathering for
%the big football game.
%%
%Online services can do a better job suggesting friends.
%%
%\jam{Too negative -$>$
%Online services can reconnect users with the community that the users neglected
%when they started spending their time online.
%}

%Geographic information is also useful for researchers.
%%
%In the past few years, there has been growth in research into topics such as
%geographical user clustering and location-based social networks.
%%
%As the accuracy of geographic information improves, it will open up new
%research opportunities.

Tempering this excitement, however is a key challenge: how to derive
high-quality location estimates for users in social media.
%
Many users choose not to reveal their location, while others may reveal their
location only using noisy or overly general descriptors (e.g., `` New York'').
%
 To tackle this challenge, one of the most popular location estimation methods
 is \textit{network-based estimation}, in which a user's location is derived
 from the known location  properties of other users nearby in the social
 network (e.g,.
%
if most of my direct friends are in San Francisco, then perhaps I am as well)
\cite{davis2011infer}, \cite{li2012towards}.
%
One of the best known network-based approaches was introduced by Facebook
researchers Backstrom et al.
%
\cite{backstrom2010find} in a comprehensive study over 2.9 million Facebook
users.
%
This Facebook study analyzed users who had posted a home street address and
found that as distance increases, the probability of friendship decreases.
%
Building on this observational study of Facebook users, they showed how the
street addresses of a user's friends could be used to predict a user's location
within 25 miles 57\% of the time.

While an encouraging first step, this Facebook approach may encounter
difficulties in location estimation when applied more broadly to other social
media services:

\begin{itemize}
\item \textit{Multiple (often imprecise) location granularities.} First, many
    users in social media reveal broad, imprecise locations (e.g., at the city
    or state level), while others provide fine-grained latitude-longitude
    information.  In particular, users are less likely to post precise
    locations such as street addresses on public websites such as Twitter.  How
    can these multiple location granularities be integrated to account for
    uncertainty at different levels?
\item \textit{Varying social ties.} Second, not all relationships in social
    media are the same.  Some ties are stronger than others, and presumably
    some ties are more revealing of a user's location.  How does this variable
    tie strength nature impact location estimation?
\item \textit{Conflicting purposes.} Finally, many social systems serve
    different purposes.  Twitter, for example, is both a social network
    connecting friends (which may tend to be local) as well as a news media
    (supporting global dissemination)  \cite{kwak2010why}.  What is the
    appropriate balance between these conflicting purposes for location
    estimation?
\end{itemize}


To address these challenges, we propose a novel network-based approach for
location estimation that
  integrates evidence of the \textit{social tie strength} between users for improved location estimation,
  naturally incorporates uncertainty across multiple location granularities,
  and can distinguish between users of the system that operate at cross-purposes.
We investigate this approach through an examination of over 100 million
geo-encoded tweets and 73 million user profiles collected from Twitter.
%
Concretely, we propose a location estimator -- FriendlyLocation -- and
investigate the relationship between the strength of the tie between a pair of
users and the distance between the pair.
%
Based on this investigation, we identify several factors such as number of
followers and how the users interact that can strongly reveal the distance
between a pair of users, and use these factors to train a tree classifier to
predict the distance between a pair of connected users.
%
We use the results of this classifier as the input to a maximum likelihood
estimator to predict a user's location.
%
We find that this proposed method significantly improves the results of
location estimation relative to the Facebook technique.
%
FriendlyLocation improves the average error distance for 80\% of Twitter users
from 41 miles to 21 miles which has great significance for augmenting
traditional social media and enriching location-based services with more
refined and accurate location estimates.

%Well-known research \cite{kwak2010why} shows that Twitter is both a social
%network and a form of news media.
%%
%Our technique allows us to distinguish between pairs of users who are likely
%to be friends in real life, and therefore connected for social reasons, and
%pairs of users who are distant and more likely to be connected for news
%distribution.

\flsec{Related work}
The study of geographical properties of online social media users has drawn
intensive attention in recent years.
%
Characterizing network properties in relation to local geography has been
studied in \cite{yardi2010tweeting}.
%
Lindqvist \cite{lindqvist2011mayor} analyzed how and why people use location
sharing services, and discussed the privacy issues related to location sharing
services.
%
User behavior with regard to the location field in Twitter user profiles has
been studied in \cite{hecht2011tweets}.

Scellato et al.  \cite{scellato2011socio} used data from three location based
social networks to investigate the relationship between distance and location,
and they showed that connections are not purely caused by geographical or
social factors.
%
They investigated two random models where they shuffle the user locations and
another where they shuffled the social connections and investigated what
happened to the user location.
%
They showed that if a user has more connections, then their friends tend to be
further away.
%
They also found that longer connections are equally likely to be part of a
social triangle as shorter connections.

Several researchers in recent years have looked into predicting user locations
in a social network based on the social graph.
%
In the largest-to-date study on this subject, Facebook researchers analyzed the
physical distance between Facebook users' social relations and utilized the
locations of a user's friends' to predict the user's geographical location
\cite{backstrom2010find}.
%
Davis et al. \cite{davis2011infer} investigated a system that predicted
location by taking a vote among the locations of the user's friends and picking
the most popular location.

An alternative to network-based approaches is \textit{content-based}, in which
a the content associated with a user either explicitly reveals location
information (e.g,. mentioning a local attraction like Disneyland) or implicitly
does so (e.g., by inferring subtle local linguistic cues to estimate a user's
location).
%
Cheng et al. \cite{cheng2010you} proposed a content-based system for locating
users on Twitter.
%
They found words that are highly concentrated in specific regions and built a
model to calculate the probability that a user lives at a location.
%
Eisenstein et al. \cite{eisenstein2010latent} proposed a functionally similar
system based on a latent variable model that predicted a user's location based
on the words in their tweets.
%
Recently, Li et al. \cite{li2012towards} proposed a system to integrate both
network and content-based estimation via a unified discriminative influence
model which combined locations that a Twitter user mentioned with the locations
of the user's followers.

In recent work \cite{sadilek2012finding}, the authors look at users who post at
least 100 geo-located tweets a month in NYC and LA.
%
They reconstruct the social graph from co-occurring geo-located tweets and used
dynamic Bayesian networks to predict where a user was at a particular moment in
time.

Finally, Cranshaw et al. \cite{cranshaw2010bridging} tackled the inverse of the
problem we are investigating: they predicted the existence of a social network
tie given precise location information from laptops and cell phones.
%
They used a collection of features of when users are co-located.
%
Together, these efforts have begun to lay a foundation for the study geo-social media.

