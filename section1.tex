\ifdefined\THESIS
    \pagestyle{plain} % No headers, just page numbers
    \pagenumbering{arabic} % Arabic numerals
    \setcounter{page}{1}
    \chapter{\uppercase {Introduction and Related Work}}
\else
\fi

\section{Introduction}

\emph{Comments are in italics.}

\emph{Fix past and present tense.}

\emph{Fix capitalization especially for Followers and Friends}

Online social networks are increasingly getting ``geographical.''
%
Millions of online social network users are voluntarily sharing their location
through location sharing services like Foursquare, short text status updates
through real-time microblogs, and photos in online image hosts.
%
This location information is increasingly being incorporated in social media
for providing localized content, location-aware recommendations, and other
geo-spatial enabled services.
%
The access to the trails and connections among millions of social media users
provides an unprecedented opportunity to study the relationship between the
geographical properties of social media users and the social relations
connecting them.

With high-quality geographic information, online services become more useful to
the local community.
%
Online services can bring to light discussions about what the city council
voted on last night.
%
Online services can identify, and react to, the crowd of people gathering for
the big football game.
%
Online services can do a better job suggesting friends.
%
\emph{Too negative -$>$
Online services can reconnect users with the community that the users neglected
when they started spending their time online.
}

Geographic information is also useful for researchers.
%
In the past few years, there has been growth in research into topics such as
geographical user clustering and location-based social networks.
%
As the accuracy of geographic information improves, it will open up new
research opportunities.

In this paper, we investigate the interplay of distance and tie strength
through an examination of over 100 million geo-encoded tweets and 73 million
user profiles collected from Twitter.
%
Concretely, we investigate the relationship between the strength of the tie
between a pair of users and the distance between the pair.
%

Based on this investigation, we identify several factors such as number of
followers and how the users interact that can strongly reveal the distance
between a pair of users, and use these factors to train a tree classifier to
predict the distance between a pair of connected users.
%
We use the results of this classifier as the input to a maximum likelihood
estimator to predict a user's location.
%
We find that this proposed method significantly improves the results of
location estimation relative to a state-of-the-art technique.
%
FriendlyLocation improves the average error distance for 80\% of Twitter users
from 41 miles to 21 miles which has great significance for augmenting
traditional social media and enriching location-based services with more
refined and accurate location estimates.

Well-known research \cite{kwak2010why} shows that Twitter is both a social
network and a form of news media.
%
Our technique allows us to distinguish between pairs of users who are likely
to be friends in real life, and therefore connected for social reasons, and
pairs of users who are distant and more likely to be connected for news
distribution.


\section{Related Work}

The study of geographical properties of online social media users has drawn
intensive attention in recent years.  Characterizing network properties in
relation to local geography has been studied in \cite{yardi2010tweeting}.
Lindqvist \cite{lindqvist2011m} analyzed how and why people use location
sharing services, and discussed the privacy issues related to location sharing
services.

Scellato et al. \cite{scellato2011socio} used data from three location based
social networks to investigate the relationship between distance and location,
and they show that connections are not purely caused by geographical or social
factors.  They investigated two random models where they shuffle the user
locations and another where they shuffle the social connections and investigate
what happens to the user location.  They show that if a user has more
connections, then their friends tend to be further away.  They also found that
longer connections are equally likely to be part of a social triangle as
shorter connections.

Several researchers in recent years have looked into predicting user locations
in a social network based on the social graph.
%
In the best-known paper on this subject, Facebook researchers analyzed
the physical distance between Facebook users' social relations and utilized the
locations of a user's friends' to predict the user's geographical location
\cite{backstrom2010find}.
%
They picked users who entered their street address in the location field of
their profile and used this to predict the location of their friends using
maximum likelihood estimation.
%
However, the information they used is generally not available outside of
Facebook, since people rarely make their street address public.
%
In a Transactions in GIS paper, Davis et all. \cite{davis2011infer}
investigated a simple system that predicted location by taking a vote among
the locations of the user's friends and picking the most popular location.
%
Recently, Li et al. \cite{li2012towards} proposed a system called the unified
discriminative influence model which combined locations that a Twitter user
mentioned with the locations of the user's followers.

Cheng et al. \cite{cheng2010you} proposed a
content-based system for locating users on Twitter. They find words that are
highly concentrated in specific regions and build a model to calculate the
probability that a user lives at a location.

Gilbert \cite{gilbert2009predicting} shows how the strength of a tie can be
predicted by interaction patterns.  They had volunteers rate the strength of
several friendships and then analyzed the activity between the users can
predict the strength of the friendship.  We argue that tie strength is also
correlated with physical proximity.

There has also been some research into specific aspects of Twitter that is
relevant to our work.  User behavior with regard to the location field in
Twitter user profiles has been studied in \cite{hecht2011tweets}.

%One of the early investigations into Twitter usage by Java \cite{java2007we}
%showed that people use twitter for chatter, conversations, sharing information,
%and reporting news.

Finally, Cranshaw et al. \cite{cranshaw2010bridging} solved the inverse of the
problem we are investigating: they predicted the existence of a social network
tie given precise location information from laptops and cell phones. They used
a collection of features of when users are co-located.  Together, these efforts
have begun to lay a foundation for the study \textit{geo-social media}.

