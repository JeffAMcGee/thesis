\flchap{Conclusion}
We demonstrated that some types of relationships tend to be closer than others,
identified some features of relationships that are correlated with physical
proximity, and used this to accurately predict the locations of users on a
social media website.
There are two directions that the future work on this project could go:
improving the results of the predictor and using the predicted locations in
other research projects.

%\jam{Do you talk about future work in a thesis?}

One way to improve this predictor is to combine tie strength and the social
graph with other factors such as the words users choose to use as described by
Cheng \cite{cheng2010you}.
It could be useful for the predictor to return not just a location, but an
estimate of the quality of the prediction.  This paper only considered
users who have a well-defined location. FriendlyLocation could identify users
who do not have meaningful locations such as people who constantly travel.
This could work well with the node locality metric which is defined by
Scellato\cite{scellato2010distance}.

Finally, high-quality geographic information opens up new avenues for research.
With geographic location of users, we can cluster users and find local
conversations.
It also allows businesses to provide hyper-local content and services.


