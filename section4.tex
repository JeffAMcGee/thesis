\chapter{\uppercase{Location Prediction}}

Based on the questions we answered in the previous section, we have enough
information to build a system for location prediction, which we call FriendlyLocation.
Since the location prediction system will be run on a large number of of users, it must be fast and scalable.
For the purposes of location prediction, we limit ourselves to just four factors per edge: the type of contact, if the target user mentioned the contact, the number of followers the contact has, and the location error of the contact.
We do not use any information about the contact's friends, followers, or who they mentioned since the amount of data this involves grows quadratically with the number of contacts.
Most of the gains that come from looking at triangles are also visible by looking at the count of followers.
We use the number of followers for a contact instead of the number of friends since it was more significant as seen in Figure \ref{fig:LocalAll}.

\section{Model}
\label{sec:model}

In this section, we build a model for the probability that a user, who we refer
to as the target user, lives at a specific location given the approximate
location of his contacts.
In the previous sections, we looked at the probability that a contact lived a certain distance from a given user.
Location prediction requires the probability that the user lives at a specific location.
Since the circumference of a circle grows linearly with the radius of a circle, the land area at a certain distance also grows.
The probability that a user lives at specific spot at a distance is proportional to the probability that she lives at that distance divided by the distance. 

While contacts with a small location error are likely to live quite near the
target user, contacts with a larger location error might live farther away. 
Instead of working measuring distance in miles, we measure distance relative
to the predicted error.
We divide the distance between the target user and their contact by the
contact's location error to create Figure \ref{fig:EdgeTypesMdist}, which shows
the probability that a contact lives at a spot.

Based on this graph, any given edge will fit into one of three categories: a local user that is closer than the predicted location error and is uniformly distributed, a nearby user whose relationship was formed by meeting in person and fits a power law with the distance, or a user whose connection is not related to geography, and is distributed based on the real-world population distribution.
For each of the four types of contacts, we plot two gray lines on the graph that represent the first two categories.
The horizontal lines are not best-fit lines; these lines represent the ratio of pairs of users with a distance less than the predicted error to the total number of pairs of that contact type. 
For example, almost 20\% of reciprocal friends live closer to each other than
the predicted location error of the contact. This is the top gray line on
\ref{fig:EdgeTypesMdist}.
The diagonal line is a best fit line for the curve where
\begin{math}1<={dist \over PLE}<10\end{math}.
We choose 10 because this is approximately where the randomly distributed friends who are not caused by proximity start to affect the distribution.
Assuming that every edge is in one and only one of these three categories, we can create a formula for the probability that a user is in a specific category:
\begin{displaymath}1=l+\int_{1}^{\infty}e^{b}x^{a}\mathrm{d}x+r\end{displaymath}
where \(l\) is the fraction of users with a distance less that the predicted
location error, \(a\) and \(b\) are the parameters of the linear regression,
and \(r\) is the fraction of users who are caused by random connections that
have no relationship to distance.
Since we know everything except \(r\) for any given curve, we can integrate and solve for \(r\):
\begin{displaymath}r=1-l+{e^{b} \over a+1}\end{displaymath}

When this model is created, it assumes that none of the randomly located users
are located near the target user.
This is reasonable when the predicted location error is small.
When the predicted error is large, the location does very little to affect the
final result since vague locations such as ``Texas'' mostly help to break ties.

\section{System}
We used this model to build a Maximum Likelihood Estimator.
FriendlyLocation trained on 27,251,927 edges from users in the training set to their contacts.
The edges were split into the 14 categories seen in Figure \ref{fig:LocalAll}, and
then further split based on the follower count. For each of these groups, it
calculated \(l\), did a least-squares fit to find \(a\) and \(b\), and used
that to calculate \(r\).
In other words, FriendlyLocation used the training data to calculate \(l\),
\(a\), \(b\), and \(r\) for each of the red boxes seen in Figure
\ref{fig:LocalAll}.

We assume that the randomly located users are uniformly distributed with
distances between 0 and \(c\). We choose \(c\) to be 2750 miles which is
approximately the distance from one corner of the U.S. to the other. The value
of this parameter has very little affect on the results except in cases where
the distance from the target user is large.
Based on the model from the previous section, the probability that a user is at
a specific distance, \(d\), is given by the following piecewise function:
\begin{displaymath}
    P(d) = FIXME
\end{displaymath}
The probability that a user is not random is divided by the PLE so that
locations with larger errors will contribute less to the final result.
The probability that a target user is at a given point can be found by
calculating the distance between the point and the location of each of the
contacts and then summing \(log(P(d))\) for each of the user's contacts.

We use this to create a simple procedure for estimating the location for a user:
\begin{enumerate}
\item Pick up to 100 of the user's contacts. (They can either be selected randomly or the more significant edges can be chosen.)
\item Geocode the location field and calculate the the predicted error for each of the contacts.
\item Ignore any contacts who have no decodable location information. Approximately one third of the contacts are ignored.
\item For each of the contacts' locations calculate the probability that the target user lives there using the maximum likelihood estimator.
\item Pick the location with the highest probability.
\end{enumerate}
The locations where the contacts live are not necessarily the maximums of the 

\section{Evaluation}
We evaluate the system against several baseline implementations, and evaluate
how the performance of the system varies with the number of contacts a user has
and how many FriendlyLocation chooses to use.

The geo-located users we used to do the evaluation are less concerned about the
privacy of their location information than the average twitter user.
As a result, they tended to give better information in the location field of
their user profile than the average twitter user.
For this evaluation we choose to ignore the contents of the geo-located user's
location fields.
It is likely that better results could be achieved in practice
by using the user's reported location a significant factor in the maximum
likelihood estimation.

We evaluated the FriendlyLocation system against 40831 of the 40861 users from the evaluation group.
Thirty users from this set were removed because none of their friends or followers had decodable locations.
(When the crawler initially created the evaluation set it filtered out users with both zero friends and zero followers.)

We investigate several implementations of the FriendlyLocation system:
\begin{description}
\item[Simple] This system treats all contacts equally---it ignores their predicted location error and any information about what type of contact it is. Users are selected randomly and it uses the same parameters for all the users.
\item[Location Error] This shows the value of calculating the median location error. Users are selected randomly.
\item[Relationship Types] This shows the value of treating contacts differently
based on the type of contact.
Users who are more likely to live near the target user, such as reciprocal
friends, are chosen first.
\item[Full] This is the system described in the previous section. It selects
users based on relationship type, uses the parameters from the training, and
uses predicted location error to evaluate the quality of the contact.
\end{description}

In addition to using the FriendlyLocation system as described in the previous
section, we created a few baseline location predictors:
\begin{description}
\item[Median] Finds the median of the latitude and the median of the longitude of 100 randomly selected contacts.
\item[Mode] Finds the most common location of 100 randomly selected contacts, and breaks ties by picking one randomly.
\item[Omniscient] Finds the contact that is closest to the user from 100 users selected by the same method used in the Full system.
\end{description}

Figure \ref{fig:FinalResults} shows a comparison to the results from the 
different predictors.
Our Full FriendlyLocation system predicts the location within 25 miles 63\% of
the time.
Unfortunately, when the predictor system is wrong, it can be very wrong; often
when it would pick not just the wrong city, but the wrong side of the
country.
Both using location error and relationship type improved the predictor.  The
simplified version of FriendlyLocation located users within 25 miles 54\% of
the time which is still better than the either of the baseline predictors.
The omniscient predictor demonstrates that there is still room for improvement
which we discuss in the final section.

We investigated the effect of increasing numbers of contacts on the quality of the results.
Before removing contacts without location information, we sorted the users into
groups based on the number of contacts the had.
Next, we ran the FriendlyLocation predictor against each of them.
Figure \ref{fig:LulResults} shows the results of prediction.
The quality of the results from the predictor was lower for users with fewer
than 50 contacts; however, as contacts increased beyond that, it had no
significant effect on the results.


The final step of the evaluation is to investigate if FriendlyLocation would work with a smaller number of profiles than 100.
For all of the users with more than 200 contacts, we sorted their contacts in
order of decreasing probability that they were local.
Then we ran FriendlyLocation on the top \(n\) contacts for \(n\in (5,10,15,25,100)\).
The results shown in Figure \ref{fig:TopResults} for 25 contacts are almost as
good as the results for 100 contacts. This means that our algorithm only needs
location information for around ten users to make reasonable predictions.


