%%%%%%%%%%%%%%%%%%%%%%%%%%%%%%%%%%%%%%%%%%%%%%%%%%%
%
%  New template code for TAMU Theses and Dissertations starting Fall 2012.
%  For more info about this template or the
%  TAMU LaTeX User's Group, see http://www.howdy.me/.
%
%  Author: Wendy Lynn Turner
%	 Version 1.0
%  Last updated 8/5/2012
%
%%%%%%%%%%%%%%%%%%%%%%%%%%%%%%%%%%%%%%%%%%%%%%%%%%%
%%%%%%%%%%%%%%%%%%%%%%%%%%%%%%%%%%%%%%%%%%%%%%%%%%%%%%%%%%%%%%%%%%%%%
%%                           ABSTRACT
%%%%%%%%%%%%%%%%%%%%%%%%%%%%%%%%%%%%%%%%%%%%%%%%%%%%%%%%%%%%%%%%%%%%%

\chapter*{ABSTRACT}
\addcontentsline{toc}{chapter}{ABSTRACT} % Needs to be set to part, so the TOC doesnt add 'CHAPTER ' prefix in the TOC.

\pagestyle{plain} % No headers, just page numbers
\pagenumbering{roman} % Roman numerals
\setcounter{page}{2}

\indent With the rise of volunteered user location information through services
like Foursquare, Google Latitude, and Facebook Places, we face unprecedented
access to the real-world connections among millions of social media users. This
location information is increasingly incorporated into social media for
providing localized content, location-aware recommendations, and other
geo-spatial enabled services.
In this paper, we investigate the interplay of
distance and tie strength through an examination of FIXME million geo-encoded
tweets collected from Twitter and FIXME million user profiles. Concretely, we
investigate the relationship between the strength of the tie between a pair of
users, and the distance between the pair. We identify several factors --
including following, mentioning, and actively engaging in conversations with
another user -- that can strongly reveal the distance between a pair of users.
We find a bimodal distribution in Twitter, with one peak around 10 miles from
people who live nearby, and another peak around 2500 miles(FIXME), further validating
Twitter's use as both a social network (with geographically nearby friends) and
as a news distribution network (with very distant relationships).

FIXME: do we really have enough of a bimodal distribution?

Based on these findings, we propose and evaluate a method for predicting
location for users based purely on an analysis of the user's social network,
which has great significance for augmenting traditional social media and
enriching location-based services with more refined and accurate location
estimates.


\pagebreak{}
