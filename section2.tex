\chapter{\uppercase {Data Collection}}

Text goes here.

\section{Crawling}
To investigate how social relation and geographical distance between the
relations correlate, we sample a dataset from Twitter.

We built a crawler to find these contacts for users who used Twitter's Location
Feature to disclose their location.
The crawler sampled FIXME geo-coded tweets by monitoring Twitter's public
Streaming API between May FIXME until May FIXME, 2012.
In the dataset, users are filtered who have strict privacy settings, have
neither friends nor followers, or posted fewer than three tweets in the sampled
dataset.
For each user with geo-coded tweets, we use the median latitude and median
longitude of locations of the user's tweets as an approximation of her home
location.
Some Twitter accounts, such as accounts that posted jobs, would
move around faster than a human could possibly move.
To account for this, the crawler calculated the distance between each tweet and
the user's home location.
The crawler ignored users if the median distance from their tweets to their
home location was greater than 50 miles.

FIXME: talk about how I picked the 2500 users

For all of the geo-located users with a home location in the US, the crawler
used Twitter's API to download the users' friends, followers, and 100 most
recent tweets.
The crawler also downloaded the profiles for up to 100 friends, followers, and
people they mention in their tweets. If they had over 100, it chose a random
sample of 100 profiles that included at most 25 of each of the four
categories listed below.

FIXME: was that right?
FIXME: the next paragraph is totally wrong.

From the profiles, the crawler kept the users with a decodable location field
and threw away the rest.
From what was left, the crawler randomly picked one
Friend, one Follower, one Just Mentioned, and up to seven reciprocal friends.
For each of these contacts, the crawler stored their friends, followers, and
100 most recent tweets.

\section{Geocoding}
FIXME: urls?
%We used Gisgraphy\footnote{\url{http://www.gisgraphy.com/}} to do geocoding.
%Gisgraphy does full-text search on the GeoNames\footnote{\url{http://www.geonames.org/}}
database using Lucene. Since
it runs locally it is not limited to a certain number of queries per day.
Gisgraphy's geocoder returns ranked results based on a full text search
over millions of geographical features such as countries, cities, and schools.

The location field on a user's profile is just a text field that asks the user
to respond to ``Where in the world are you?''.
Responses vary from precise latitude and longitude coordinates entered
automatically by smart phone apps to jokes and nonsense.
Our system does some preprocessing before sending user-submitted locations to
the geocoder.
First, it uses a regular expression to find latitude and longitude coordinates.
These are treated as if they were a unique type of location returned by the
geocoder.
Occasionally, users would put two locations separated by a slash, dash or a
conjunction.
If the geocoder did not return any results for a user, our system tried to
geocode both locations and returned the first location that the geocoder
understood.

Some locations are significantly more useful than others.
For example, even though Rhode Island and Montana are both states with around
one million people, Rhode Island is smaller, and as a result, much more useful
in estimating the location of a user.
To make the results of the geocoder more useful, we devised a method to
estimate the accuracy of a location returned by the geocoder.
Because users in the US had friends outside of the US, FriendlyLocation needs
to estimate the quality of
locations outside of the US as well.
As a result, we trained the geocoder trained on geographic data from around the
world.

\begin{table}[t]
\centering
\caption{Example Median Location Errors (miles)}
\begin{tabular}{l r r} 
Location&Number of Users&Median Error\\ \hline
``Bronx''&53&4\\
``New York''&1264&174\\
``Pluto''&11&7843\\ \hline
\\
Place Type&Number of Users&Median Error\\ \hline
(latitude, longitude)&25216&3\\
A City&7128&6\\
An Airport&117&31\\
A Country&37&3935\\
\hline\end{tabular}
\label{tab:MedianLocErr}
\end{table}

For the users in the geocoding training and evaluation groups, we have both the
home location based on geo-located tweets, and the free-response location field
that the geocoder was able to decode.
We define the \textbf{location error} to be the great circle distance between a
user's home location and the location returned from the geocoder.
The location error can vary from less than a mile to over ten thousand miles.
Each user in the geocoding training group was geocoded and grouped based on the
exact location returned by the geocoder.
For the FIXME locations that had at least three users, we calculated the median
location error for that location.
All the users who were at locations with only one or two users were grouped
according to the type of location returned by the geocoder.
We calculated the median error for each location type.
The median is more appropriate than the average or standard deviation because
those metrics are strongly affected by large outliers.
We choose to make the cutoff three because that is the smallest value where the
median is not just an average.
Table~\ref{tab:MedianLocErr} shows the median location error for a few example
locations and location types.

This gives us a method to estimate the quality of a coordinate returned by
Gisgraphy.
We define the \textbf{predicted location error (PLE)} for a given location to
be the median location error if it is one of the FIXME locations that had three
users; otherwise, it is the median location error for the location's type.
For example one user had ``PDX,OR'' in his location field. Gisgraphy identifies
this as ``Portland International Airport''. Since it is not one of the most
common locations, its PLE is determined by its place type to be 31 miles as
seen in Table~\ref{tab:MedianLocErr}.

FIXME: do we even want this figure?
Figure~\ref{fig:DiffGnpGps} shows the result of using the PLE to ignore users
who have low quality locations such as ``Pluto''.
The solid black line represents the normal results of geocoding where the
location error is less than 1000 miles for 88\% of the users.
The cyan line shows the results of removing locations with a PLE greater than
or equal to 1000 miles.
After removing only 6\% of the users, 93\% of them are within 1000 miles.
